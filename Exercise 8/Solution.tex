\documentclass{article}
\usepackage[utf8]{inputenc}
\usepackage{bussproofs}
\usepackage{amssymb}
\usepackage{latexsym}
\usepackage{geometry}
\geometry{
 a4paper,
 total={170mm,257mm},
 left=20mm,
 right=20mm,
 top=20mm,
 }

\title{Introduction to Informatics 2}
\author{Mathias Quintero}

\begin{document}

\maketitle

\section{Exercise 12.1.}

\subsection{12.1.1}
\label{sub:12.1.1}

We start with the statement.

\begin{equation}
  e = []
\end{equation}

We can infer as an absolute truth that [ ] terminates. Meaning:

\begin{equation}
  $$
  \begin{prooftree}
      \AxiomC{$true$}
      \UnaryInfC{$ [ \: ] \quad terminates$}
  \end{prooftree}
  $$
\end{equation}

Therefore:

\begin{equation}
  $$
  \begin{prooftree}
      \AxiomC{$e = [ \: ]$}
      \UnaryInfC{$ e \quad terminates$}
  \end{prooftree}
  $$
\end{equation}

We define:

\begin{equation}
  let \: {e}_{match} \:  e = \: match\: e\: with \: [ \: ] \: ->\: {e}_{1} \: | \: x::xs \: ->\: {e}_{2}
\end{equation}

Meaning:

\begin{equation}
  { \Pi  }_{ e } =
  $$
  \begin{prooftree}
      \AxiomC{$ let \: {e}_{match} \:  e = \: match\: e\: with \: [ \: ] \: ->\: {e}_{1} \: | \: x::xs \: ->\: {e}_{2} $}
      \UnaryInfC{$ {e}_{match} \:  e \: \Rightarrow match\: e\: with \: [ \: ] \: ->\: {e}_{1} \: | \: x::xs \: ->\: {e}_{2}$}
  \end{prooftree}
  $$
\end{equation}

And prove:

\begin{equation}
  $$
  \begin{prooftree}
      \AxiomC{$e = [ \: ]$}
      \AxiomC{${ \Pi  }_{ e }$}
      \BinaryInfC{$ {e}_{match} \:  e \: \Rightarrow \: {e}_{1}$}
      \UnaryInfC{$ match\: e\: with \: [ \: ] \: ->\: {e}_{1} \: | \: x::xs \: ->\: {e}_{2} \Rightarrow  \: {e}_{1}$}
  \end{prooftree}
  $$
\end{equation}

Since ${e}_{match}$ returns ${e}_{1}$ directly when handed e as a parameter we know that the equality is absolute.

\subsection{12.1.2}
\label{sub:12.1.2}

We start with the assumption that e terminates.

Meaning that anything that equals e will also terminate:

\begin{equation}
  { \Pi  }_{ t } =
  $$
  \begin{prooftree}
      \AxiomC{$e \quad terminates$}
      \AxiomC{$ e = e'::e''$}
      \BinaryInfC{$ e'::e'' \quad terminates $}
  \end{prooftree}
  $$
\end{equation}

Therefore we can actually use the value for e infer a match, using the information from 12.1.1.

This means that $ {e}_{match} \:  e$ terminates when the matched expression terminates and therefore equality is present.

Since the Pattern Matching names our head and tail x and xs we have to change those in ${e}_{2}$

\begin{equation}
  $$
  \begin{prooftree}
      \AxiomC{${ \Pi  }_{ e }$}
      \AxiomC{${ \Pi  }_{ t }$}
      \BinaryInfC{$ {e}_{match} \:  e \: \Rightarrow \: {e}_{2}[e'/x,e''/xs] $}
      \UnaryInfC{$ match\: e\: with \: [ \: ] \: ->\: {e}_{1} \: | \: x::xs \: ->\: {e}_{2} \Rightarrow  \: {e}_{2}[e'/x,e''/xs]$}
  \end{prooftree}
  $$
\end{equation}

\subsection{12.1.3}
\label{sub:12.1.3}

For this one we define:

\begin{equation}
  let \: {f}_{1} =\: fun\: x \: -> \: match\: x\: with \: [ \: ] \: ->\: 1 \: | \: x::xs \: ->\: 42
\end{equation}

and:

\begin{equation}
  let \: {f}_{2} =\:fun \: x \: ->\: 42
\end{equation}

We therefore can assume:

\begin{equation}
  { \Pi  }_{ {f}_{1} } =
  $$
  \begin{prooftree}
      \AxiomC{$let \: {f}_{1} =\: fun\: x \: -> \: match\: x\: with \: [ \: ] \: ->\: 1 \: | \: x::xs \: ->\: 42$}
      \UnaryInfC{${f}_{1} \: x \: \Rightarrow \: match\: x\: with \: [ \: ] \: ->\: 1 \: | \: x::xs \: ->\: 42$}
      \AxiomC{$ x \: \Rightarrow h::t$}
      \BinaryInfC{${f}_{1} \: x \: \Rightarrow 42$}
      \AxiomC{$[1;2;3] \: \Rightarrow \: 1::[2;3]$}
      \BinaryInfC{${f}_{1} \: [1;2;3] \: \Rightarrow 42$}
  \end{prooftree}
  $$
\end{equation}

Since [1;2;3] is a constant value and therefore always terminates.

In a similar fashion we can also determine a value for ${f}_{2}$ 5:

\begin{equation}
  { \Pi  }_{ {f}_{2} } =
  $$
  \begin{prooftree}
      \AxiomC{$let \: {f}_{2} =\: fun\: x \: -> \: match\: x\: with \: [ \: ] \: ->\: 1 \: | \: x::xs \: ->\: 42$}
      \UnaryInfC{${f}_{2} \: x \: \Rightarrow \: 42$}
      \AxiomC{$5 \: \Rightarrow \: 5$}
      \BinaryInfC{${f}_{2} \: 5 \: \Rightarrow 42$}
  \end{prooftree}
  $$
\end{equation}

And we can as a result determine:

\begin{equation}
  $$
  \begin{prooftree}
      \AxiomC{${ \Pi  }_{ {f}_{1} }$}
      \AxiomC{${ \Pi  }_{ {f}_{2} }$}
      \AxiomC{$42 \quad terminates$}
      \TrinaryInfC{${f}_{1} \: [1;2;3] = {f}_{2} \: 5$}
  \end{prooftree}
  $$
\end{equation}

\end{document}
