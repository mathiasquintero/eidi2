\documentclass{article}
\usepackage[utf8]{inputenc}
\usepackage{bussproofs}
\usepackage{amssymb}
\usepackage{latexsym}
\usepackage{geometry}
\geometry{
 a4paper,
 total={170mm,257mm},
 left=20mm,
 right=20mm,
 top=20mm,
 }

\title{Introduction to Informatics 2}
\author{Mathias Quintero}

\begin{document}

\maketitle

\section{Exercise 11.1.}

We start with the following functions:

\begin{equation}
  let\:f\:= \:fun\: x\: ->\: x+1*2
\end{equation}
\begin{equation}
  let\:g\:= \:fun\: x\: ->\: x+42
\end{equation}
\begin{equation}
  let\:h\:= \:3
\end{equation}

For everything we're about to prove we create the following statements (to make our job much easier):

\begin{equation}
    { \Pi  }_{ f } = $$
    \begin{prooftree}
        \AxiomC{$let\: f\: =\: fun\: x\: ->\: x\: +\: 1\: \times \: 2$}
        \UnaryInfC{$f\: x\: \Rightarrow \: x\: +\: 1\: \times \: 2$}
        \AxiomC{$1\: \times \: 2\: \Rightarrow \: 2$}
        \BinaryInfC{$f\: x\: \Rightarrow \: x\: +\: 2$}
    \end{prooftree}
    $$
\end{equation}

\begin{equation}
    { \Pi  }_{ g } = $$
    \begin{prooftree}
        \AxiomC{$let\: g \: =\: fun\: x\: ->\: x\: +\: 42$}
        \UnaryInfC{$g\: x\: \Rightarrow \: x\: +\: 42$}
    \end{prooftree}
    $$
\end{equation}

\begin{equation}
    { \Pi  }_{ h } = $$
    \begin{prooftree}
        \AxiomC{$let\: h\: =\: 3$}
        \AxiomC{$3 \: \Rightarrow \: 3$}
        \BinaryInfC{$g\: x\: \Rightarrow \: x\: +\: 42$}
    \end{prooftree}
    $$
\end{equation}

\subsection{Match}
\label{sub:Match}
We define let ${ e  }_{ match } \: x = match \: x \: with \: [ ] \:->\: 0\: |\: x::xs\: ->\: f\: x$
. Following that:
\begin{equation}
    { \Pi  }_{ { e  }_{ match } } = $$
    \begin{prooftree}
        \AxiomC{$let \: { e  }_{ match } \: x = match \: x \: with \: [ ] \:->\: 0\: |\: x::xs\: ->\: f\: x $}
        \UnaryInfC{$ { e  }_{ match } \: x \Rightarrow \: match \: x \: with \: [ ] \:->\: 0\: |\: x::xs\: ->\: f\: x $}
    \end{prooftree}
    $$
\end{equation}

So we infer:
\begin{prooftree}
    \AxiomC{${ \Pi  }_{ { e  }_{ match } }$}
    \AxiomC{$ n \: \Rightarrow \: h::t $}
    \BinaryInfC{$ { e  }_{ match } \: n \: \Rightarrow \: f h$}
    \AxiomC{$ [1;2;3] \: \Rightarrow \: 1::[2;3]$}
    \BinaryInfC{$ { e  }_{ match } \: [1;2;3] \: \Rightarrow \: f 1 $}
    \AxiomC{$ { \Pi  }_{ f } $}
    \AxiomC{$ 1 + 2 \: \Rigtharrow \Rightarrow \: 3 $}
    \TrinaryInfC{$ { e  }_{ match } \: [1;2;3] \: \Rightarrow \: 3 $}
\end{prooftree}
$$
\square
$$

\subsection{Internal Var}
\label{sub:Internal Var}
We define let k =$\: let\: x\: =\: 5\: in\: g\: x\: +\: h$
so that:
\begin{prooftree}
        \AxiomC{$let\: k\: =\: let\: x\: =\: 5\: in\: g\: x\: +\: h$}
        \UnaryInfC{$ k \: \Rightarrow \: g \: 5 \: + \: h $}
        \AxiomC{${ \Pi  }_{ h }$}
        \BinaryInfC{$k \: \Rightarrow \: g \: 5 \: + 3$}
        \AxiomC{${ \Pi  }_{ g }$}
        \AxiomC{$5 + 42 + 3 \Rightarrow 50$}
        \TrinaryInfC{$k \: \Rightarrow \: 50$}
    \end{prooftree}
    $$
\square
$$

\section{Exercise 11.2}

For this exercise we concentrate on proving if staments terminate or not.
For this I won't be using the inference writing since using induction for this is much quicker and the slides of Prof. Seidl show him using this method more often.

\subsection{Does square terminate?}
\label{sub:Does square terminate?}
We have to prove that for the following statements:
\begin{equation}
  let\:rec\: doit\: a\: x\: =\: if\: x = 0\: then\: a\: else\: doit\: (a+2*x-1)\: (x-1)
\end{equation}
\begin{equation}
  let\: square\: x\: =\: doit\: 0\: x
\end{equation}
the function square x terminates for every positive number x.

We will do this by method of induction.
For this we will be proving the following:
\begin{equation}
  doit\: y\: n\: =\: { f }_{ n }(y)
\end{equation}
with
\begin{equation}
  \forall n \in \mathbb{N}\cup \{ 0\}  : \forall y \in \mathbb{R} : \exists { f }_{ n }(y)
\end{equation}
With this in mind we start by proving both for n = 0
By the definition of doit we get:
\begin{equation}
  doit\: y\: 0 \: = \: y
\end{equation}
So we can confirm that:
\begin{equation}
  { f }_{ 0 }(y) = y
\end{equation}
Meaning we've succesfully proven the statement for n = 0.
Now we will assume that for a specific n our conditions apply, so that
\begin{equation}
  doit\: y\: n\: =\: { f }_{ n }(y)
\end{equation}
will hold a value for every y, and will be proving based on this assumption that it will also hold a value for n+1.
\begin{equation}
  doit\: y\: (n+1)\: =\: doit\: (y+2n+1)\: n\: =\: { f }_{ n }(y+2n+1)
\end{equation}
Therefore we can conclude that:
\begin{equation}
  { f }_{ n+1 }(y) = { f }_{ n }(y+2n+1)
\end{equation}
And since ${ f }_{ n }(y)$ is defined for every y, so is ${ f }_{ n+1 }(y)$. Including 0.
Our programm therefore terminates for every positive number n, since they all return a value.
$$
\square
$$


\end{document}
